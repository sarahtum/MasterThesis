\documentclass{article}
%\usepackage[latin1]{inputenc}% erm\"oglich die direkte Eingabe der Umlaute 
%\usepackage[T1]{fontenc} % das Trennen der Umlaute
%\usepackage{ngerman} % hiermit werden deutsche Bezeichnungen genutzt und 
                     % die W\"orter werden anhand der neue Rechtschreibung 
		     % automatisch getrennt.  
\usepackage{amsmath}	% for formulas, matrices,
\usepackage{amssymb} % for mathbb	
\usepackage{enumitem} % for hiding bullets in itemize environment
\usepackage[margin=1in]{geometry} % for definition of margin     
\usepackage{hyperref} % for \ref, \eqref, \url(?)
\usepackage[dvipsnames]{xcolor} % for textcolors
\usepackage{cite} % for citations
\usepackage{verbatim} % for comment environment
\title{\textbf{Fragen vor Treffen 12.06.}}
%\author{}
\date{}
\begin{document}
\maketitle
\paragraph{Paper allgemein}
\begin{itemize}
\item Sager: "first discretize, then optimize" $\rightarrow$ large number of integer variables\\
\textit{Werden wir auch so machen, wenn Dynamik (also ODE) mitmodelliert wird, da indirekte Methoden zu aufwendig}
\item Optimal control? (H\"angt davon ab, ob geeignete Kostenfunktion)\\
\textit{Mit Roboterdynamik kriegen wir ein OCP. Ob wir das so im Abstract erwähnen, entscheiden wir morgen.}
\end{itemize}

\paragraph{Paper W\"achter}
\begin{itemize}
\item keine so tollen Laufzeiten, \texttt{BONMIN} (als Heuristik) viel schneller\\
\textit{Wir betrachten kein riesiges Problem, können außerdem Größe über Feinheit der Diskretisierung selbst etwas steuern.}
\item \"Anderung \texttt{BONMIN/IPOPT} zu anderen Algos m\"oglich, ohne alles neu zu programmieren?\\
\textit{Der Plan ist ohnehin, alles selber zu programmieren. Da \texttt{BONMIN} und \texttt{IPOPT} open source, sollte das machbar sein. Sebastian hat W\"achter-\texttt{AMPL}-Implementierung nicht open source gefunden.}
\item \textit{Nicht so gut an W\"achter: wenig Beweise (nur convergence analysis). Gleiches Problem mit Kesavan (da viele Beweise "wegzitiert", Mathematik ist also vorhanden)}
\item \textit{Gut: viele numerische Beispiele, kann man in Arbeit als Vergleichsreferenz verwenden ("Haben Algorithmus implementiert, der Ergebnisse von W\"achter-Paper f\"ur diese Beispiele reproduziert", auch wenn es dann f\"ur Robotik-Problem nicht funktioniert)}
\end{itemize}

\paragraph{Modell}
\begin{itemize}
\item Constraints, objective?\\
\textit{Haben uns zweierlei MINLPs angeschaut: einmal statisches MINLP, einmal dynamisches MIOCP (siehe n\"achste Seite)}
\item wo treten integer-Variablen auf?\\
\textit{Nur "Objekt gegriffen bzw. nicht gegriffen", i.e. nur eine diskrete Variable \textbf{pro Zeitschritt}}
\item separabel? 
\textcolor{green}{\textit{noch ungekl\"art}}
\item Gr\"o\ss{}enordnung?
\textcolor{green}{\textit{noch ungekl\"art}}
\item inklusive Dynamik?
\textit{Das ist definitiv das Ziel}
\end{itemize}

\paragraph{Erwartungen}
\begin{itemize}
\item Laufzeiten?
\textcolor{green}{\textit{noch ungekl\"art}}
\item nur vereinfachtes Modell?
\textit{Ja}
\item besser: optimal, Dynamik\\
\textit{Und vielleicht lernen wir aus dem Modell am Ende noch was, neue Ideen, evtl wiederverwendbar als Teil eines Learning-Ansatzes. Dauerhaftes Rumschrauben an Parametern und Heuristiken-Basteln kann keine L\"osung sein.}
\end{itemize}


\newpage
\paragraph{@Modell:}
\begin{enumerate}
\item \textit{Statisches} MINLP:
	\begin{itemize}
	\item state $x$ enth\"alt Roboterzustand und Objektzustand
	\item binary variable $w$ beschreibt, in welcher Mannigfaltigkeit wir uns befinden
	\item $F_1$ beschreibt zul\"assige Konfigurationen in placement manifold ($x$ zul\"assig, wenn $F_1(x)\leq 0$),\\ $F_2$ f\"ur grasp manifold
	\item Diskretisierung $\tau_0,...,\tau_N$
	\item Modell:
	\begin{align}
	find(x,w) \text{ s.t.}&\label{eq:obj}\\
	 &w(\tau) F_1(x(\tau)) + (1-w(\tau))F_2(x(\tau)) &&\leq 0 &\forall \tau\label{eq:zulässig}\\
	 &\big(w(\tau+1)-w(\tau)\big)\big(x(\tau+1)-x(\tau)\big) &&= 0 &\forall \tau\label{eq:transition}\\
	 &\|x(\tau + 1)-x(\tau)\| &&\leq \alpha \label{eq:abstand}
	\end{align} 
		\begin{itemize}
		\item @\eqref{eq:obj}: hier evtl. \"ahnlich gewichtete objective function wie Philipp sie verwendet
		\item @\eqref{eq:zulässig}: Alle Zust\"ande zu allen Diskretisierungszeitpunkten sind zul\"assig.
		\item @\eqref{eq:transition}: Wenn $w$ das Vorzeichen wechselt, muss Position gleich bleiben.
		\item @\eqref{eq:abstand}: damit Lösung nicht nur sechs Punkte enth\"alt, i.e. ausreichend viele "samples"
		\end{itemize}
	\end{itemize}
\item \textit{Dynamisches} Optimal Control Problem (so k\"onnen wir es immer aufschreiben, die Frage ist, wie kompliziert es ist)
	\begin{itemize}
	\item Massenmatrix $M$, \"andert sich durch Greifen des Objekts, da dieses Masse hat
	\item Korriolis- und Gravitationskr\"afte stecken in $\rho$
	\item Torques $T$
	\item Roboterdynamik:
		\begin{align}
		T = M(q) \ddot{q} + \rho(q,\dot{q}) \label{eq:dynamic}
		\end{align}
		\begin{itemize}
		\item kann man umschreiben als $1^{st}$ order ODE (via $\begin{pmatrix} q\\\dot{q}\end{pmatrix}$)
		\item ODE diskretisieren: $\hat{x}_{t+1}= \Phi(x_t,u_t)$ s.t. $x_{t+1}-\hat{x}_{t+1}=0$ mit Integrator $\Phi$ (nichtlinear)
		\item verwende hierzu expliziten Euler, Runge-Kutta-Verfahren,...
		\item L\"osungsvektor $(x_0,u_0,x_1,u_1,...,x_T,u_T)$, sodass \#integer Variablen = 1 $\cdot$ \#Zeitschritte\\
		$\rightarrow$ evtl doch in Sager-Paper f\"ur gro\ss{}e Anzahlen an integer Variablen (allerdings wird unser $T$ nicht riesig, i.e. nicht in $\mathcal{O}(10^3) - \mathcal{O}(10^4)$)
		\end{itemize}
	\end{itemize}
\item Vergleich: 
	\begin{itemize}
	\item Anzahl integer variables bei beiden \"ahnlich
	\item Nichtlinearit\"aten in statischem Modell evtl. sch\"ner, Integrator in dynamischen Modell sicher h\"asslich
	\end{itemize}
\end{enumerate}

\newpage
\paragraph{Zwei S\"aulen:} Arbeit braucht roten Faden, insbesondere Verknüpfung Heuristik und MINLP. (Hauptinteresse Siemens: Beschleunigung durch Heuristik f\"ur Manipulationsplanung. Hauptinteresse Sebastian/Uni: Modellierung MINLP)
\begin{enumerate}
\item Vergleich  erweiterte Heuristik mit MINLP Problem \eqref{eq:obj} - \eqref{eq:abstand}.
	\begin{itemize}
	\item L\"osung des MINLP Problems oben ist mindestens so gut wie das, was Heuristik liefert. \\$\rightarrow$ Vergleich m\"oglich
	\item evtl. kann MINLP sogar helfen, Heuristik zu entwerfen
	\end{itemize}
\item Vergleich MIOCP/MINLP (inklusive Dynamik \eqref{eq:dynamic}) mit Philipps Ansatz + dynamisches Post-Processing
	\begin{itemize}
	\item im Idealfall: Haben wir ein Verfahren, das Modell inklusive Dynamik \eqref{eq:dynamic} l\"ost, k\"onnen wir damit auch Modell \eqref{eq:obj} - \eqref{eq:abstand} l\"osen, sodass S\"aule 1 nicht mehr viel Aufwand
	\end{itemize}
\end{enumerate}
\end{document}