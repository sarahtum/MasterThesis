\documentclass{article}

%\usepackage[latin1]{inputenc}% erm\"oglich die direkte Eingabe der Umlaute 
%\usepackage[T1]{fontenc} % das Trennen der Umlaute
%\usepackage{ngerman} % hiermit werden deutsche Bezeichnungen genutzt und 
                     % die W\"orter werden anhand der neue Rechtschreibung 
		     % automatisch getrennt.  
\usepackage{amsmath}	% for formulas
\usepackage{amssymb} % for mathbb	
\usepackage[margin=1in]{geometry} % for definition of margin     
\usepackage{hyperref}
\usepackage{color}
\usepackage{cite}
\usepackage{graphicx}
\title{\textbf{Basics about Optimal control}}
%\author{}
\date{}

\begin{document}

\maketitle

\section{Basic Definitions}
\begin{itemize}
\item Optimal Control is process of determining \textit{control and state trajectories} for dynamic system over period of time minimizing some cost function
\item General case according to \url{http://www.scholarpedia.org/article/Optimal_control}\\
\textbf{\textcolor{blue}{Bolza Problem}}:
\begin{align}
\min_{u} ~~~ & J = \varphi(x(t_f)) + \int_{t_0}^{t_f} L(x(t), u(t),t) ~~ dt \label{eq:obj}\\
&\text{s.t. } \dot{x}(t) = f(x(t),u(t),t),  x(t_0) = x_0 \label{eq:ODE}
\end{align}
where $[t_0,t_f]$ is time interval of interest, $x:[t_0,t_f]\rightarrow \mathbb{R}^{n_x}$ is state vector, $u:[t_0,t_f]\rightarrow \mathbb{R}^{n_u}$ is control vector, $\varphi$ is terminal cost function, $L$ is intermediate cost function, $f$ is vector field.

Equation \eqref{eq:ODE} represents dynamics of the system and initial state condition.

\item \textbf{\textcolor{blue}{Mayer Problem}}: if $L(x,u,t)=0$
\item \textbf{\textcolor{blue}{Lagrange Problem}}: if $\varphi(x(t_f))=0$
\end{itemize}

\section{Catchwords to look up}
\begin{itemize}
\item Dynamic programming (Richard Bellmann)
\end{itemize}




\end{document}