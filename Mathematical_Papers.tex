\documentclass{article}

%\usepackage[latin1]{inputenc}% erm\"oglich die direkte Eingabe der Umlaute 
%\usepackage[T1]{fontenc} % das Trennen der Umlaute
%\usepackage{ngerman} % hiermit werden deutsche Bezeichnungen genutzt und 
                     % die W\"orter werden anhand der neue Rechtschreibung 
		     % automatisch getrennt.  
\usepackage{amsmath}	% for formulas
\usepackage{amssymb} % for mathbb	
\usepackage[margin=1in]{geometry} % for definition of margin     
\usepackage{hyperref}
\usepackage{color}
\usepackage{cite}
\title{\textbf{Overview over Mathematical MINLP Papers}}
%\author{}
%\date{\today}

\begin{document}

\maketitle

\section{TODO}
\subsection{Leute}
\begin{itemize}
\item Prof. Sebastian Sager (Uni Magdeburg)
\item Christian Kirches
\item Bock-Gruppe Heidelberg
\item oft gelesen: Grossmann
\item McCormick (spatial branching, oft zitiert von \cite{burer2012non})
\end{itemize}

\subsection{Themen}
\begin{itemize}
\item Outer Approximation
\item outer relaxation (Sebastian)
\item outer convexification
\item hybrid methods (see references in \cite{burer2012non})
\end{itemize}

\subsection{Paper}
\begin{itemize}
\item Outer approximation algorithms for separable nonconvex mixed-integer nonlinear programs
\item Bonami: An algorithmic framework for convex MINLP
\item R. Karuppiah, I.E. Grossman, A Lagrangian based branch-and-cut algorithm
for global optimization of nonconvex mixed-integer nonlinear programs
with decomposable structures
\item C. D’Ambrosio, J. Lee, A. W\"achter, An algorithmic framework for MINLP with
separable non-convexity
\item . Leyffer, Integrating SQP and branch-and-bound for mixed integer
nonlinear programming (use exact algo for convex MINLP as heuristic for non-convex)
\item G. Nannicini, P. Belotti, Rounding-based heuristics for nonconvex MINLPs,
Math. Program. Comput. (heuristic for non-convex MINLPs solving alternating sequence of NLPs and MILPs)
\item literaturliste von \cite{burer2012non}
\item The integer approximation error in mixed-integer optimal control (see below)
\item and paper which is mentioned below
\item Kirches, Bock, Leyffer: Modeling Mixed-Integer Constrained
Optimal Control Problems in AMPL
\item Kirches, Bock, Sager: Mixed-integer NMPC for predictive cruise control of heavy-duty trucks
\item Ruiz, Grossmann: Global Optimization of Non-convex Generalized
Disjunctive Programs: A Review on Relaxations and Solution Methods (maybe similar methods can be used for non-convex MINLPs?)
\item Grossmann: Cutting planes for improved global logic-based outer approximation
for the synthesis of process networks (similar to non-convex outer approximation for MINLPs of Kesvan et al, but for non-convex generalized disjunctive programs instead)
\item Ambrosio: Mixed integer nonlinear programming tools: an updated practical overview
\item Bonami, Hijazi, Ouorou: An Outer-Inner Approximation for separable MINLP
\end{itemize}

\subsection{Fragen an Sebastian}
\begin{itemize}
\item Sager: methods for MINLP that come from MIOCPs (mixed integer optimal control problems) by "first discretize, then optimize". Therefore large number of integer variables s.t. standard techniques like nonlinear Branch-and-Bound and Outer Approximation fail. This is not what we are planning to do?!
\item MIOCPs?
\item outer approximation: usually only for convex MINLPs. Trotzdem so ein Paper nehmen oder explizit was zu non-convex suchen (was schwierig ist)?
\item our problem: separable functions?
\end{itemize}

\newpage
\section{Branch and Bound}
\subsection{Combinatorial integral approximation [2010, Sager, Jung, Kirches]}
\begin{itemize}
\item decomposition of MINLP into NLP and MILP
\item bounds for gap between MILP ad MINLP solution
\item propose tailored Branch and Bound strategy
\end{itemize}

\subsection{The Lagrangian Relaxation for the Combinatorial Integral Approximation Problem [2012, Jung, Sager]}
\begin{itemize}
\item continues previous paper
\item study properties of Lagrangian of MILP
\item prove polynomial runtime in special cases
\end{itemize}

\subsection{Solving Mixed-Integer Nonlinear Programs by QP-Diving [2012, Mahajan, Leyffer, Kirches]}
\textcolor{red}{PROBLEM:} convexity
\begin{itemize}
\item new variant of Branch-and-Bound
\item rather than solving expensive NLP at each node of Branch-and-Bound tree, solve quadratic approximation (QP approximation) at every node
\item use WP solves that can be hot-started
\item global convergence properties for convex MINLPs
\end{itemize}

\newpage
\section{Outer approximation}

\subsection{FilMINT: An Outer-Approximation-Based Solver for Nonlinear Mixed Integer Programs [Abhishek, Leyffer, Linderoth, 2008]}
\begin{itemize}
\item solver capable of solving MINLPs at cost which is a small multiple of cost of a comparable MILP
\item algorithm used: \textcolor{blue}{LP/NLP-based branch-and-bound} (LP/NLP-BB) \cite{quesada1992lp}
\begin{itemize}
\item similar to outer-approximation, but:
\item instead of solving alternating sequence of MILP master problems, NLP subproblems: interrupt solution of MILP master whenever new integer assignment is found, then solve a NLP subproblem
\item (as usual:) solution of this subproblem provides new outer approximation added to master MILP, solving updated MILP master continues
\item thus: only single branch-and-cut tree is created and searched
\end{itemize}
\item 
\item 
\end{itemize}



\subsection{FilMINT: An Outer Approximation-Based Solver for Convex Mixed-Integer Nonlinear Programs [Abhishek, Leyffer, Linderoth, 2010]}
\textcolor{red}{PROBLEM:} convexity
\begin{itemize}
\item solver for convex MINLPs implementing linearization-based algorithm
\item avoids complete re-solution of master MILP by dding new linearizations at open nodes of branch-and-bound tree whenever integer solution is found
\item FilMINT combines MINTO branch-and-cut framework for MILP (allows to use cutting planes, primal heuristics etc.)  with filterSQP for NLPs (active set solver allowing warm-starting, taking advantage of good initial primal and dual iterates)
\end{itemize}





\newpage
\subsection{Outer approximation algorithms for separable nonconvex mixed-integer nonlinear programs [Kesvan, Allgor, Gatzke, Barton, 2004]}
see \cite{kesavan2004outer}\\
\textcolor{green}{Promising}\\
\textcolor{red}{PROBLEM:} old\\
\textbf{What?} 
\begin{itemize}
\item  decomposition approach constisting of solving alternating sequence of relaxed master MILPs and two NLPs
\item use Primal Bounding Problems, i.e. a convex NLP solved at each iteration to derive valid outer approximations of nonconvex functions in continous space
\item convergence and optimality properties proven
\item for non-convex MINLPs: recently a couple of algorithms based ond branch-and-bound strategies have been developed (e.g. branch-and-reduce, $\alpha$BB, spatial BB...). All based on solving lower bounding relaxed master problem, upper bounding primal problem at each node of BB tree. Only differ in formulation of master and primal problem and heuristic braching strategy.
\item this paper: these convex under-estimators employed by BB-approaches may also be exploited to develop decomposition algorithms for non-convex MINLPs
\item subproblems used in algorithm: (derived in section 2) 	
	\begin{itemize}
	\item Lower Bounding Convex MINLP: solution yields lower bound to global solution of original problem ($P$)
	\item Master Problem: MILP, solution yields lower bound to global solution of ($P$)
	\item Relaxed Master Problem: MILP, solution is lower bound on subset not yet explored
	\item Primal Problem: nonconvex NLP obtained by fixing binary variables ($y$) in ($P$). Any feasible solution yields upper bound to ($P$)
	\item Primal Bounding Problem: convex NLP, solution tighter lower bound to primal problem for each fixed binary realization $y$ than the bound provided by relaxed master problem generating $y$
	\end{itemize}
\item section 3: algorithms presented
\end{itemize}

\newpage
\section{neither Branch-and-Bound nor outer approximation}
\subsection{The integer approximation error in mixed-integer optimal control [2010, Sager, Bock]}
\begin{itemize}
\item Problem: MINLP that come from discretizing MIOCPs cannot efficiently be solved using nonlinear BnB or outer approximation
\item therefore Sager, Reinelt, Bock suggested another approach in "Direct methods with maximal lower bound for mixed-integer optimal control problems" [2009]
\item on nonlinear optimal control problems with integer restrictions
\item theorem: solution of relaxes and convexified problem can be approximated arbitrarily close by solution fulfilling the integer requirements
\item constructive way to obtain integer solution with bound on performance loss
\end{itemize}

\subsection{Non-convex mixed-integer nonlinear programming: A survey [Burer, Letchford, 2012]}
see \cite{burer2012non}\\
\textcolor{green}{Promising}\\
\textbf{What?} Summary of approaches for solving non-convex MINLPs
\begin{itemize}
\item non-convex MINLP are challenging because  continous relaxation in general not convex
\item key concepts forming main ingredients of all existing eaxct algorithms for non-convex MINLPs
	\begin{itemize}
	\item \emph{Under- and Over-estimators}
		\begin{itemize}
		\item as solving continous relaxation still difficult, need a further relaxation step
		\item replace each non-convex function $f^j(x,y)$ with convex under-estimating function $g^j(x,y)$ (i.e. $g^j(x,y) \leq f^j(x,y)$)
		\item or replace each non-convex function $f^j(x,y)$ by new variable $z^j$ acting as a placeholder and add constraints enforcing that $z^j$ is approximately  $f^j(x,y)$, e.g. $z^j \geq g^j(x,y)$
		\item same works for concave over-estimating functions or use linear under-/overestimating functions when want to use LP solver
		\item for specific functions: convex and concave envelopes (tightest possible convex under-estimator/concave over-estimator)
		\item for examples see references in \cite{burer2012non}, page 99.
		\end{itemize}
	\item \emph{Separable functions}
		\begin{itemize}
		\item function $f(x,y)$ separable if $f(x,y) = \sum_{i=1}^{n_1}g(x_i) + \sum_{j=1}^{n_2}h(y_j)$
		\item sum of individual under-estimators is under-estimator for $f$
		\item approximate non-convex MINLP by MILP by: approximate each individual function with piecewise linear function, introduce continuous variables $g_i, h_i$ representing values of those functions, introduce binary variables for each piece of some linear function, introduce further binary variables with linear constraints to ensure that $g_i, h_i$ take correct values
		\end{itemize}
	\item \emph{Factorization}
		\begin{itemize}
		\item introduce new variables and constraints s.t. resulting MINLP involves functions of simpler form
		\item then find under-/over-estimators for simple functions
		\end{itemize}
	\item \emph{Branching: standard and spatial}
		\begin{itemize}
		\item standard branching: branch on integer-constrained variable
		\item spatial branching: branch by partitioning domain of continous variables and additionally replace original under- and over-estimators with stronger one, taking advantage of reduced domain
		\end{itemize}
	\end{itemize}
\item algorithms for general case of non-convex MINLPs
	\begin{itemize}
	\item \emph{Spatial branch-and-bound}
		\begin{itemize}
		\item arrange subproblems in tree-structure
		\item prune subproblems if (i) feasible but its cost under relaxed objective equals its true cost, (ii) associated lower bound no better than bst upper bound so far, (ii) infeasible
		\end{itemize}
	\item \emph{Branch-and-reduce}
		\begin{itemize}
		\item improved version of spatial branch-and-bound, attempting to reduce variable domains by more than what results of branching
		\item add two operations:\\ (i) before solving subproblem, check its constraints whther domain of any variable can be reduced without losing any \emph{feasible} solutions;\\ (ii) after solving subproblem, check whether domain of any variable can be reduced without losing any \emph{optimal} solutions
		\item afterwards generate better under-estimator, tighten constraints, possibly improved lower bounds
		\item drastic decrease in size of tree
		\item usally use LP relaxations (rather than convex programming relaxations)
		\end{itemize}
	\item \emph{$\alpha$-branch-and-bound}
		\begin{itemize}
		\item based on general technique for constructing under-estimators
		\item all functions need to be twice differentiable
		\item no additional variables needed, no need of factorization
		\item one needs a general convex programming solver, not only LP solver
		\end{itemize}
	\item \emph{Conversion to MILP}
		\begin{itemize}
		\item factorize problem, approximate resulting separable MINLP by MILP (see above), solve resulting MILP
		\item leads to sets of binary variables with special structure, use specialized branching rule
		\end{itemize}
	\item \emph{Heuristics}
		\begin{itemize}
		\item sometimes possible to convert exact algorithms for convex MINLPs into heuristics for non-convex MINLPs, e.g. see \cite{leyffer2001integrating} or \cite{nannicini2012rounding}
		\end{itemize}
	\end{itemize}
\item algorithms for special case of all non-linear functions are quadratic
	\begin{itemize}
	\item describes reformulation-linearization technique
	\end{itemize}
\item Software 
	\begin{itemize}
	\item packages solving non-convex MINLPs to proven optimality 
	\item packages which can be used to find heuristic solutions for non-convex MINLPs: BONMIN, DICOPT, etc.
	\end{itemize}	 
\end{itemize}

\newpage
\section{General}
\subsection{Mixed-integer nonlinear optimization [Belotti, Kirches, Leyffer, Majahan, 2013]}
\begin{itemize}
\item models and applications of MINLP
\item state of the art methods
\item both convex and non-convex MINLP
\item non-convex MINLP
\begin{itemize}
\item challenging because even after relaxing the integer variables to continuity the feasible region is in general non-convex, i.e. many local minimma
\item also challenging as linearizations do not yield valid lower bounds any more
\item approaches: piecewise linear approximations, generic strategies for obtaining convex relaxations, spatial branch-and-bound, under-/overestimation of nonconvex functions
\end{itemize}
\item review heuristic techniques obtaining good feasible solutions
\end{itemize}

\subsection{Numerical methods for mixed-integer optimal control problems [Sager, PhD-Thesis, 2005]}
see \cite{sager2005numerical}
\begin{itemize}
\item Chapter 1: MIOCP, Chapter 2: Optimal control, Chapter 3: MINLP
\item 
\end{itemize}

\subsection{Fast Numerical Methods for Mixed-Integer Nonlinear Model-Predictive Control [Kirches, PhD-Thesis, 2010]}

\subsection{Mixed integer nonlinear programming tools: an updated practical overview [Ambrosio, 2013]}
\url{https://link.springer.com/article/10.1007/s10479-012-1272-5}

\bibliography{../library}{}
\bibliographystyle{plain}
\end{document}

