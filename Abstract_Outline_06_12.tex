\documentclass{article}
%\usepackage[latin1]{inputenc}% erm\"oglich die direkte Eingabe der Umlaute 
%\usepackage[T1]{fontenc} % das Trennen der Umlaute
%\usepackage{ngerman} % hiermit werden deutsche Bezeichnungen genutzt und 
                     % die W\"orter werden anhand der neue Rechtschreibung 
		     % automatisch getrennt.  
\usepackage{amsmath}	% for formulas
\usepackage{amssymb} % for mathbb	
\usepackage{enumitem} % for hiding bullets in itemize environment
\usepackage[margin=1in]{geometry} % for definition of margin     
\usepackage{hyperref} % for \ref, \eqref, \url(?)
\usepackage[dvipsnames]{xcolor} % for textcolors
\usepackage{cite} % for citations
\usepackage{verbatim} % for comment environment
\title{\textbf{Outline for Abstract}}
%\author{}
\date{}
\begin{document}

\maketitle
\section{Abstract}

\paragraph{Content:}
\begin{itemize}
\item General introduction:
	\begin{itemize}
	\item explain problem of motion planning and manipulation planning
	\item goal of thesis: model this using MINLP, solved as in paper, comparison to state-of-the-art sampling heuristic
	\end{itemize}

\item Part I: Mathematics: MINLP
	\begin{itemize}
	\item basics in MINLP
	\item one approach for non-convex MINLP with certain separability properties: paper of W\"achter
	\item possible modifications (different solvers than BONMIN, problem specific)
	\end{itemize}

\item Part II: Mathematics: modeling as MINLP
	\begin{itemize}
	\item modeling our problem as MINLP, applying algorithm of paper 
	\item Types of constraints, robot dynamics?	
	\end{itemize}		
	
\item Part III: Robotic: Extension of ellipsoidal heuristic
	\begin{itemize}
	\item describe current sampling procedure
	\item describe existing heuristic for \textit{motion planning}
	\item describe Philipps approach how to extend sampling to manipulation planning
	\item identify difficulties for extension to \textit{manipulation planning}
	\item identify sections where heuristic can be easily adapted
	\item hopefully come up with ideas how to extend heuristic
	\end{itemize}
\end{itemize}

\paragraph{Expectations:}
\begin{itemize}
\item MINLP:
	\begin{itemize}[label={}]
	\item \textcolor{red}{$-$} MINLP won't be competitive in terms of runtime
	\item \textcolor{red}{$-$} MINLP only for simplified version of problem
	\item \textcolor{green}{+} optimal solution (even globally)
	\item \textcolor{green}{+} able to embed robot dynamics (not like sampling approach: optimal solution w.r.t. objective function not containing dynamics)
	\end{itemize}
\item heuristic for manipulation planning:
	\begin{itemize}
	\item 
	\end{itemize}
\end{itemize}
\newpage
\section{Model}
\paragraph{Objective Function:}
\begin{itemize}
\item minimize completion time
\item 
\end{itemize}



\paragraph{Constraints:}
\begin{itemize}
\item Robot Dynamics
	\begin{itemize}
	\item $1^{st}$ order ODE
	\item states: joint angles, joint velocities (i.e. \texttt{MultiDOFState})
	\item control: joint torques
	\end{itemize}
\item collision-free
\item upper and lower bounds on states
\item interaction with pick-and-place object
\item 
\item 
\end{itemize}

\paragraph{Variables:} integer variables?
\begin{itemize}
\item 
\item 
\end{itemize}

\paragraph{Separability:} ?

\end{document}