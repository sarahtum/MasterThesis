\documentclass{article}
%\usepackage[latin1]{inputenc}% erm\"oglich die direkte Eingabe der Umlaute 
%\usepackage[T1]{fontenc} % das Trennen der Umlaute
%\usepackage{ngerman} % hiermit werden deutsche Bezeichnungen genutzt und 
                     % die W\"orter werden anhand der neue Rechtschreibung 
		     % automatisch getrennt.  
\usepackage{amsmath}	% for formulas
\usepackage{amssymb} % for mathbb	
\usepackage{enumitem} % for hiding bullets in itemize environment
\usepackage[margin=1in]{geometry} % for definition of margin     
\usepackage{hyperref} % for \ref, \eqref, \url(?)
\usepackage[dvipsnames]{xcolor} % for textcolors
\usepackage{cite} % for citations
\usepackage{verbatim} % for comment environment
\title{\textbf{Outline for Abstract}}
%\author{}
\date{}
\begin{document}

\maketitle
\paragraph{Soll enthalten:}
\begin{itemize}
\item Quellen (inklusive Literaturverzeichnis)
\item Aufgabenstellung
\item m\"ogliche Arbeitsrichtungen
\end{itemize}

\section{Current state of work}
\begin{itemize}
\item describe current sampling procedure for motion planning
\item heuristic is necessary for making it faster, describe Ellipsoid heuristic
\item describe Philipps approach how to extend sampling to manipulation planning, see \cite{schmitt2017manipulation}
\item \textit{Wie direkt reinschreiben, dass wir versuchen wollen, diese Heuristik f\"ur manipulation planning auszuweiten?}
\end{itemize}
\textit{Fehlt: Einleitender Satz}

Most state-of-the-art algorithms solving the problem of motion planning use sampling-based approaches which construct a topological graph (the "roadmap") the node set of which represents a set of random valid robot configurations. To find the desired trajectory, a search is performed on this graph connecting the initial configuration to some goal configuration. In order to increase the efficiency of this sampling procedure, Gammell et al. \cite{gammell2014informed} suggest a heuristic describing those samples that can possibly improve the previous solution which highly improves the convergence rate.

When we consider manipulation tasks for a robot, we do not only need to determine the motion itself but also how to grasp and release an object. \textit{(sehr nah an Philipps Text)}. Solving grasps and placements of some objects and the motion of the robot simultaneuosly is considered as \textit{manipulation planning}.
In \cite{schmitt2017manipulation}, Schmitt et al. present a sampling-based manipulation planner which extends the classical sampling-based roadmap planners. Extending the idea of \cite{gammell2014informed} to manipulation planning, which particularly requires establishing a heuristic for finding good transition configurations between grasps and placements, is expected to increase the runtime significantly.


\section{Basic motivation MINLP}
\begin{itemize}
\item integer variable as an object is in grasp or not
\item reference to robotics paper \cite{peng2005coordinating} using MINLP
\end{itemize}

All feasible and collision-free robot and object configuration corresponding to some specific placement or grasp of the object (i.e. some attachment of the object with its static environment or the robot's gripper) form a submanifold. Modeling the state whether the object is in some grasp or not as a binary decision variable, the feasibility, collision-avoidance and dynamic constraints for each grasp and placement can be combined to formulate the manipulation problem as a mixed integer nonlinear program (MINLP). 

A similar approach is proposed in \cite{peng2005coordinating} where determining the control inputs for the collision-free coordination of multiple robots, minimizing the total completion time, is solved using a MINLP.

\section{reference paper/MINLP/MIOCP}
\begin{itemize}
\item describe basic ideas of paper
\item describe possible extensions (e.g. use different solvers when implemented)
\end{itemize}
\section{Possible Directions}
Explain possible MINLP models that could be set up, possible comparisons with current state of work

S\"aule 2: One could possibly set up a MIOCP/MINLP modeling a simplified version of the problem of manipulation planning which is then solved following the approach in [reference paper]. Subsequently, the resulting trajectories and completion times could be compared to the outcomes of the manipulation planner as in \cite{schmitt2017manipulation} which is usually combined with different versions of  post-processing, transferring the geometric path into a trajectory satisfying the dynamic constraints.

S\"aule 1: Another idea would be to consider a static MINLP modeling the search for valid robot and object configurations in all submanifolds, including possible transitions between them, starting with the initial configuration and ending in some  goal configuration. An optimal solution of such a model could again be compared to the outcome of the manipulation planner (omitting the dynamic post-processing). One can also think of various comparisons, depending on whether the manipulation planning algorithm is improved by some extended heuristic or not.

One might even be able to use the just mentioned static MINLP to gain valuable insights into "good" transition regions between the manifolds which might help in establishing an extension of the sampling heuristic.

\section{Goals of the Thesis}
Define expectations 
\begin{itemize}[label={}]
	\item \textcolor{red}{$-$} MINLP won't be competitive in terms of runtime
	\item \textcolor{red}{$-$} MINLP only for simplified version of problem
	\item \textcolor{green}{+} optimal solution (even globally)
	\item \textcolor{green}{+} able to embed robot dynamics (not like sampling approach: optimal solution w.r.t. objective function not containing dynamics)
\end{itemize}
and what must be fulfilled for thesis to be "very good"

\bibliography{../library}{}
\bibliographystyle{plain}
\end{document}