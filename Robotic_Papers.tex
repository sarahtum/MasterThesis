\documentclass{article}
%\usepackage[latin1]{inputenc}% erm\"oglich die direkte Eingabe der Umlaute 
%\usepackage[T1]{fontenc} % das Trennen der Umlaute
%\usepackage{ngerman} % hiermit werden deutsche Bezeichnungen genutzt und 
                     % die W\"orter werden anhand der neue Rechtschreibung 
		     % automatisch getrennt.  
\usepackage{amsmath}	% for formulas
\usepackage{amssymb} % for mathbb	
\usepackage[margin=1in]{geometry} % for definition of margin     
\usepackage{hyperref}
\usepackage{color}
\usepackage{cite}
\title{\textbf{Overview over Robotics MINLP Papers}}
%\author{}
\date{}
\begin{document}

\maketitle
%\tableofcontents

\section{Coordinating Multiple Double Integrator Robots on a Roadmap: Convexity and Global Optimality [Peng, Akella, 2005]}
see \cite{peng2005coordinating}
\begin{itemize}
\item find global minimum time control for collision-free coordination of multiple robots
\item double integrator dynamics
\item state constraints, control constraints
\item initially assume each robot's path is speicfied
\item formulate MINLP, show convexity of constraints under certain assumptions
	\begin{itemize}
	\item MINLP from page 2765 on
	\item objective: minimize completion time
	\item constraints: 
		\begin{itemize}
		\item linear completion time, traversale time, collision avoidance
		\item nonlinear minimum and maximum traversal time constraints
		\item convex feasible velocity constraints
		\end{itemize}				
	\end{itemize}
\item extend results to task of coordinating robots on given roadmap with multiple candidate paths for each robots
	\begin{itemize}
	\item select each robot's traversal path
	\item generate its continouos velocity profile (satisfiying dynamics cnstraints, collision-free, globally minimize completion time)
	\item see section VII in \cite{peng2005coordinating}
	\item for each path combination (i.e. valid selections of paths for set of robots) we have corresponding MINLP problem
	\item model roadmap coordination problem as a single MINLP by using additional binary variables
	\end{itemize}
\end{itemize}


\bibliography{../library}{}
\bibliographystyle{plain}
\end{document}
